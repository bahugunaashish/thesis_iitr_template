\cleardoublepage
\chapter{\textbf{Introduction}}\label{ch:chp1}
\section{File sytem}
There is main thesis.tex file which is the master file that is to be compiled by the text editor. all the secondary files are called with in this file. there are mainly two folders where different chapters are created first is '/prelim/' which contains the abstract, Acknowledgement, copyright, Declaration, Dedication, which could be modify by the user of this template accordingly.

Next is '/mainpage/' folder which contains chapters of the thesis, for refrence 2 chapter files are created namely chp1, chp2, user can add more chapter files as per requirement and add them to the main 'thesis.tex' file accordingly similar to chp1 and chp2.

\section{Chapters}
 to add chapters in the file add a tex file in the 'mainpage' folder with file name starting with 'chp' such as for chapter 1 the file name should be chp1.tex.

 the format of chp file should be as follows
 %

 the \verb|\chapter| will create the chapter number in grey colour as shown in this chapter mark as 1. the \verb|\textbf{}| make the content in bold which is placed inside the curly brackets. \verb|\label{}| command create a label for the heading which can be called in any chapter or section for reference using \verb|\ref{label_of_heading}|.
 %
 \section{Section}
 to add section in the chapter \verb|\section{}| command is used.



 \subsection{subsection}
 To add subsection \verb|\subsection{}| command is used.

 

 \subsubsection{subsubsection}
 To add subsubsection \verb|\subsubsection{}| command is used.
 
	 %
  \section{Figures}
  To add the figures in the section, first the file should add in the \verb|/image/| folder. for including the image file in the section, figure environment can be used to add single image, the code are shown below.
 
 %
 the command \verb|\begin{figure} …\end{figure}| creates a figure environment. \verb|\centering| command will align the figure in the center. command \verb|\includegraphics[width=1\linewidth]|
 \\ \verb|{image/image_file_name}| add the figure in the section. further, size of figure can be modify changing the value of width, you could change the value from 1 to desirable value depending upon your requirements. command \verb|\caption{}| create the caption of the figure below the image. command \verb|\label{}| create the label of figure which can be used to cite the figure in the text.
	\begin{figure}[htb!]
		\centering
		\includegraphics[width=1\linewidth]{image/iitrlogo.jpg} 
		\caption{\centering Logo of IIT Roorkee.} \label{Fig 1.1}
	\end{figure}
 %
 
 further if the requirement is to add multiple figure together with one common caption, the following command can be used.
%
\begin{figure}[htb!]
	\centering
    \begin{subfigure}[b]{0.4\textwidth}
        \includegraphics[width=\linewidth]{image/iitrlogo.jpg} 
        \caption{}
        \label{Fig 1.2a}
    \end{subfigure}
	\centering
    \begin{subfigure}[b]{0.4\textwidth}
        \includegraphics[width=\linewidth]{image/iitrlogo.jpg} 
        \caption{}
        \label{Fig 1.2b}
    \end{subfigure}
    \begin{subfigure}[b]{0.4\textwidth}
        \includegraphics[width=\linewidth]{image/iitrlogo.jpg} 
        \caption{}
        \label{Fig 1.2c}
    \end{subfigure}
	\caption{\centering Logo of IIT Roorkee. (a) Logo of IIT Roorkee. (b) Logo of IIT Roorkee. (c) Logo of IIT Roorkee.} \label{Fig 1.2}
\end{figure}
%
\section{Tables}
there could be to type two type of the table used in the thesis. (1) normal portrait table (small table width--wise) (2) landscape table (large table width--wise).

in portrait mode there are two type of tables.
first is table 1.1
\begin{table}
\centering
	 \caption{\centering The elastic Properties used in the 
  homogeneous model.}\label{Table 1.1}\\
%	 \resizebox{\columnwidth}{!}
{\begin{tabular}{c l c c c c c}
			\hline
			\textbf{S. No.} & \textbf{H1} & 
   \textbf{H2} & \multicolumn{1}{l}{\textbf{H3}} & 
   \multicolumn{1}{l}{\textbf{H4}} & \multicolumn{1}{l}
   {\textbf{H5}} & \textbf{H6} \\
			\hline
			1 & 11 & xc &  hj & hj & ij & jk \\
			2&  hg & jk  &nm  &jk & df & jk \\
			\hline
	\end{tabular}}
	\end{table}
which could be add with following code.
if table overflow from the page in this example add a code \verb| \resizebox{\columnwidth}{!}| before \verb|{\begin{tabular}{c l c c c c c}| in above table enviroment.
%

another example is shown in table 1.2, which could be generated using following code.

\begin{table}[hbt!]
	\centering
	\caption{\centering Table long vertical.}\\ \label{Table 5.4}
	\footnotesize
	\begin{tabular}{l c c c c c c c c c c c c}
			\hline
			\multirow{2}{*}{\textbf{H1}} & \multirow{2}{*}{\textbf{H2}} & \multirow{2}{*}{\textbf{H3}} & \multicolumn{4}{l}{\textbf{H4}}                                              & \multicolumn{4}{l}{\textbf{H5}}                                            & \multicolumn{2}{l}{H6}\\ \cline{4-13} 
    &   &   & h1& h2& h3& h4& h5& h6& h7& h8& h9& h10  \\ \hline
     DELH &  28.48 &  77.13 &  -1.38  &  0.46   &  -2.15 &  2.82 &  -0.60  & 0.32                           &  0.72  &  0.05 &  -0.92 &  0.67  \\
 IITK &  26.51 &  80.23 &  0.27 &  0.53   &  2.64  &  3.13 &  -1.67  & 1.41 &  0.06  &  0.55 &  -0.26 &  -0.49 \\
 LUCK &  26.89 &  80.94 &  2.43   &  1.71   &  -3.02 &  2.82 &  -1.56 & 2.28   &  -1.59 &  1.39 &  0.72  &  -0.20  \\
 BHUP &  25.27 &  82.99 &  0.08   &  0.07   &  0.17  &  0.44 &  -1.43  & 1.58   &  0.91  &  0.29 &  0.15  &  0.61\\
 KHAV &  23.92 &  69.77 &  1.18   &  2.57   &  2.46  &  3.54 &  -0.78  & 0.61   &  0.76  &  1.84 &  -1.39 &  -1.08 \\
 RADP &  23.82 &  71.62 &  0.57   &  1.16   &  0.38  &  0.85 &  1.96   & 0.23   &  0.51  &  0.98 &  1.73  &  -0.47 \\
 BELP &  23.87 &  70.8  &  0.19   &  1.82   &  0.66  &  1.55 &  0.39   & 2.02   &  0.50  &  1.39 &  -1.63 &  -0.89 \\
 MABU &  24.65 &  72.78 &  -1.67  &  1.72   &  1.35  &  2.09 &  3.16   & 3.11   &  6.36  &  7.12 &  0.05  &  -0.75 \\
 
\hline
    \end{tabular}
    
    Note: A note regarding table could be put here.
\end{table} 
%
another type of table could be large table width--wise can be created using \verb|\begin{landscape} \begin{longtable} \caption{..} ......... \end{longtable} \end{landscape}| enviroment. which can be further used as follows for a table shown in the below.
\begin{landscape}
 		\centering
 	
 		\begin{longtable}{>\centering m{0.3cm} >\centering m{3cm} >\centering m{1.2cm} >\centering m{1.5cm} >\centering m{2.5cm} >\centering m{2.5cm} >\centering m{2.5cm} >\centering m{2.5cm} c }
 		\caption{Example of Long landscape table.}
 		\label{Table 2.1}
 		\hline
 		S.No. & H1 & H2 & H3 & H4 & H5 & H6 & H7 & H8 \\
 		\hline
 		\endfirsthead
 		
 		\hline
 		\multicolumn{9}{r}{\textit{Continued...}}\\
 			\endfoot
 		
 		\multicolumn{9}{c}{\tablename~\thetable--\textit{Continued...}}\\
 	
 		\hline
 		S.No. & H1 & H2 & H3 & H4 & H5 & H6 & H7 & H8 \\
 		\hline		\endhead \endlastfoot
 		
 		1.&	2&-	&-&-&-&-&-&-\\
 		2.&	1&-	&-&3&-&-&-&-\\
 		3.&	-&-	&-&-&-&-&-&-\\	
 		4.&	-&-	&-&-&4&-&-&-\\
 		5.&	-&-	&-&-&-&-&-&-\\
 		6.&	-&-	&-&-&-&-&-&-\\
 		7.&	-&-	&-&-&-&-&-&-\\
 		8.&	-&-	&-&-&-&-&-&-\\
 		9.&	-&-	&-&-&-&-&-&-\\
 		10.&-&-	&-&-&-&-&-&-\\
 		11.&-&-	&-&-&-&-&-&-\\
 		12.&-&-	&-&-&-&-&-&-\\
 		13.&-&-	&-&-&-&-&-&-\\	
 		14.&-&-	&-&-&-&-&-&-\\	
 		15&	-&-	&-&-&-&-&-&-\\
 		16.&-&-	&-&-&-&-&-&-\\
 		17.&-&-	&-&-&-&-&-&-\\
            18.&-&-	&-&-&-&-&-&-\\
            19.&-&-	&-&-&-&-&-&-\\
            21.&-&-	&-&-&-&-&-&-\\
            22.&-&-	&-&-&-&-&-&-\\
            23.&-&-	&-&-&-&-&-&-\\
            24.&-&-	&-&-&-&-&-&-\\
            25.&-&-	&-&-&-&-&-&-\\
            26.&-&-	&-&-&-&-&-&-\\
            27.&-&-	&-&-&-&-&-&-\\
            28.&-&-	&-&-&-&-&-&-\\
            29.&-&-	&-&-&-&-&-&-\\
            30.&-&-	&-&-&-&-&-&-\\
 		\hline
 		\end{longtable}
   Note: Could be here.
\end{landscape}
%
\section{Itemize/enumerate}
To itemize \verb|\begin{itemize}...\end{itemize}| environment is used as follows.
\begin{itemize}
    \item first item
    \item second item
    \item third item
    \begin{itemize}
        \item first item
        \item second item
        \item third item
     \end{itemize}
\end{itemize}

%
whereas enumeration can be incorporated using \verb|\begin{enumerate} ...|

\verb|\end{enumerate}| environment.
\begin{enumerate}
    \item first 
    \item second
    \item third
    \begin{enumerate}
        \item first 
        \item second
        \item third
    \end{enumerate}
\end{enumerate}
%
%
\section{Equations}
simplest of equation can be written using \verb|\begin{equation*}....\end{equation*}| command
as :
 \begin{equation*}\label{}
	   x+y=z
\end{equation*}

the * create equation without the equation number.

to create the equation with the number * should be removed such as
\begin{equation}\label{}
	   x+y=z
\end{equation}
%
%
further combined equation such as:
\begin{equation}\label{1.2}
	\centering
	\begin{cases}
    \begin{aligned}
        U(x,y) = \alpha_1 + \alpha_2x + \alpha_3y + \alpha_4xy \\
	    V(x,y) = \alpha_5 + \alpha_6x + \alpha_7y + \alpha_8xy
    \end{aligned} 
    \end{cases}
\end{equation}
%
the matrix equationis written in \verb|\begin{gather}...\end{gather}| environment as:

\begin{gather}\label{eq3.4}
        \begin{Bmatrix}	
		u_1\\ u_2 \\u_3 \\u_4
		\end{Bmatrix}
		=
		\begin{bmatrix}
		1& x_1     & y_1 & x_1y_1 \\
		1& x_2     & y_2 & x_2y_2 \\
		1& x_3     & y_3 & x_3y_3 \\
		1& x_4     & y_4 & x_4y_4 \\
		\end{bmatrix}
		\begin{Bmatrix}
	    \alpha_1 \\ \alpha_2 \\ \alpha_3 \\ \alpha_4
	    \end{Bmatrix}
\end{gather}
%
A partial differential equation can be written as
\begin{equation}
     \frac{\partial U}{\partial Y} = \frac{\partial u}{\partial y} + \frac{\partial v}{\partial x}
\end{equation}
%
Integral with summation equation can be written as:
\begin{equation}
     \sigma (t) = \int^t_0 E_\circ \frac{\partial\epsilon (s)}{\partial s} ds + \int^t_0 \, \sum^N_{i=1}E_i \, \exp (-\frac{t-s}{\tau_{t,i}})  \, \frac{\partial\epsilon (s)}{\partial s} ds\\
\end{equation}